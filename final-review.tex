\documentclass[12pt]{article}
\usepackage[top = 1in,
            left = 1in,
            right = 1in,
            bottom = 1in]{geometry}
\usepackage{amsmath}
\usepackage{stackengine}
\usepackage{amsfonts}
\usepackage{minted}
\usepackage{graphicx}
\newcounter{counter}
\setcounter{counter}{1}
\newcommand*{\question}[1]{
            \textbf{\thecounter. #1} \hfill
            \addtocounter{counter}{1}
            \\ \\
            }
\setlength\parindent{0pt}
\begin{document}
\begin{center}
    \section*{EECS 334 Final Review}
\end{center}
\question{Compute the resultant irradiance of coherent and incoherent superposition of waves of the same frequency.}
First lets define irradiance. Irradiance is the flux of radiant energy unit area (normal to the direction of flow of radiant energy through a medium), or more colloquially, the fact of shining brightly. Basically, irradiance is a way to measure the energy from light that hits a certain area for a certain amount of time.
\\ \\ 
Lets define coherent waves; two waves sources are coherent if their \textbf{frequency} and \textbf{waveform} are identical.
\begin{center}
    \includegraphics*[scale = 1]{imgs/coherent-wave.png}    
\end{center}

Lastly, the definition of incoherent waves is two or more waves that have don't have the same frequency and phase.


\end{document}