\documentclass[10pt]{article}
\usepackage[top = 1in,
            left = 1in,
            right = 1in,
            bottom = 1in]{geometry}
\usepackage{amsmath}
\usepackage{stackengine}
\usepackage{amsfonts}
\usepackage{minted}
\usepackage{graphicx}
\newcounter{counter}
\setcounter{counter}{1}
\newcommand*{\question}[1]{
            \textbf{\thecounter. #1} \hfill
            \addtocounter{counter}{1}
            \\ \\
            }
\setlength\parindent{0pt}
\begin{document}
\begin{center}
    \section*{EECS 334 Final Review}
\end{center}
\question{Compute the resultant irradiance of coherent and incoherent superposition of waves of the same frequency.}
First lets define irradiance. Irradiance is the flux of radiant energy unit area (normal to the direction of flow of radiant energy through a medium), or more colloquially, the fact of shining brightly. Basically, irradiance is a way to measure the energy from light that hits a certain area for a certain amount of time.
\\ \\ 
Lets define coherent waves; two waves sources are coherent if their \textbf{frequency} and \textbf{waveform} are identical.
\begin{center}
    \includegraphics*[scale = 1]{imgs/coherent-wave.png}    
\end{center}

Lastly, the definition of incoherent waves is two or more waves that have don't have the same frequency and phase.

\begin{center}
    \includegraphics*[scale = .5]{imgs/phase-difference.png}
\end{center}

\newpage 

So, what this statement is asking us is to add these two waves, which we can do because of superposition! Superposition is the fact that if we have $\lambda_1$ and $\lambda_2$ we can add them together. If they interfere \textbf{constructively} then they will add into a bigger wave than the original, if interfere \textbf{destructive}, then the resulting wave is smaller.

\begin{center}
    \includegraphics*[scale = .5]{imgs/superposition.png}
\end{center}
We can say that the resulting wave due to the two path differences is
\[E_R = E_1 + E_2 = E_{01}\cos(\alpha_1 - wt) + E_{02}\cos(\alpha_2 - wt)\]
\begin{itemize}
    \item If $(\alpha_2 - \alpha_1) = m2\pi$, where $m$ is an integer, then the waves are in-step (the peaks and the troughs arrive at a point $P$ at the same time). This is \textbf{constructive interference}.
    \item $(\alpha_2 - \alpha_1) = (2m+1)\pi$, then the waves are out of step, resulting in \textbf{destructive interference}.
\end{itemize}

\textit{Note: When you hear out of phase, it usually means the wave source is shifted by $\pi$.}
\\ \\
A real life example of destructive interference. When someone is wearing noise cancelling headphones, the noise coming in has its own wave, but the headphones will send a single back cancelling out the the original wave. This is using the idea of two waves being completely destructive!
\begin{center}
    \includegraphics*[scale = .25]{imgs/wavelength-fact.jpeg}
\end{center}

\newpage

If we keep shifting to the right by an interger value, we will keep on getting constructive interference! Look at the diagram below. Wavelength 2 is red and wavelength 1 is black.

\begin{center}
    \includegraphics*[scale = .35]{imgs/interger-difference.jpeg}
\end{center}



\end{document}