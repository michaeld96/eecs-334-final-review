\documentclass[10pt]{article}
\usepackage[top = 1in,
            left = 1in,
            right = 1in,
            bottom = 1in]{geometry}
\usepackage{amsmath}
\usepackage{stackengine}
\usepackage{amsfonts}
\usepackage{minted}
\usepackage{graphicx}
\newcounter{counter}
\setcounter{counter}{1}
\newcommand*{\question}[1]{
            \textbf{\thecounter. #1} \hfill
            \addtocounter{counter}{1}
            \\ \\
            }
\setlength\parindent{0pt}
\begin{document}
\begin{center}
    \section*{EECS 334 Final Review}
\end{center}
\question{Compute the resultant irradiance of coherent and incoherent superposition of waves of the same frequency.}
First lets define irradiance. Irradiance is the flux of radiant energy unit area (normal to the direction of flow of radiant energy through a medium), or more colloquially, the fact of shining brightly. Basically, irradiance is a way to measure the energy from light that hits a certain area for a certain amount of time.
\\ \\ 
Lets define coherent waves; two waves sources are coherent if their \textbf{frequency} and \textbf{waveform} are identical.
\begin{center}
    \includegraphics*[scale = 1]{imgs/coherent-wave.png}    
\end{center}

Lastly, the definition of incoherent waves is two or more waves that have don't have the same frequency and phase.

\begin{center}
    \includegraphics*[scale = .5]{imgs/phase-difference.png}
\end{center}

\newpage 

So, what this statement is asking us is to add these two waves, which we can do because of superposition! Superposition is the fact that if we have $\lambda_1$ and $\lambda_2$ we can add them together. If they interfere \textbf{constructively} then they will add into a bigger wave than the original, if interfere \textbf{destructive}, then the resulting wave is smaller.

\begin{center}
    \includegraphics*[scale = .5]{imgs/superposition.png}
\end{center}
We can say that the resulting wave due to the two path differences is
\[E_R = E_1 + E_2 = E_{01}\cos(\alpha_1 - wt) + E_{02}\cos(\alpha_2 - wt)\]
\begin{itemize}
    \item If $(\alpha_2 - \alpha_1) = m2\pi$, where $m$ is an integer, then the waves are in-step (the peaks and the troughs arrive at a point $P$ at the same time). This is \textbf{constructive interference}.
    \item $(\alpha_2 - \alpha_1) = (2m+1)\pi$, then the waves are out of step, resulting in \textbf{destructive interference}.
\end{itemize}

\textit{Note: When you hear out of phase, it usually means the wave source is shifted by $\pi$.}
\\ \\
A real life example of destructive interference. When someone is wearing noise cancelling headphones, the noise coming in has its own wave, but the headphones will send a single back cancelling out the the original wave. This is using the idea of two waves being completely destructive!
\begin{center}
    \includegraphics*[scale = .25]{imgs/wavelength-fact.jpeg}
\end{center}

\newpage

If we keep shifting to the right by an interger value, we will keep on getting constructive interference! Look at the diagram below. Wavelength 2 is red and wavelength 1 is black.

\begin{center}
    \includegraphics*[scale = .35]{imgs/interger-difference.jpeg}
\end{center}

Now let's try an example.\\ \\
\textbf{Example:} \\ \\ 
Determine the result of the superposition of the following harmonic waves:
\textit{Note: $\frac{\pi}{3}$ is the phase shift, $\omega$ is the frequency.}
\[E_1 = 7 \cos(\frac{\pi}{3} - \omega t), E_2 = 12 \sin(\frac{\pi}{4} - \omega t), \text{, and } E_3 = 20 \cos(\frac{\pi}{5} - \omega t)\]

The first thing we need to do is make all the phase angles consistent. We do this by making them all the same trig function.

\begin{center}
    \includegraphics*[scale = .2]{imgs/superposition-example.jpeg}
\end{center}

Now that we have the amplitude, lets het the phase angle. We can achieve this by 
\[\tan(\alpha) = \frac{9.33}{28.165},\qquad \alpha = 0.32\text{rad}\]

Therefore, the resulting wave is 

\[\boxed{E_R = 29.67 \cos(0.32 - \omega t)} \]

\textit{Note: We "got rid" of $\omega$ and $t$ from the calculations because we are taking a screen shot of the waves and finding their superposition.}
\\ \\ 
\question{Explain the properties of standing waves and of the modes of a cavity}

The definition of a \textbf{standing wave} is when we confine a wave in a medium that has boundaries, and then this wave will reflect at the boundary, and the wave will overlap with itself. \\\\

The reason why we care about these \textbf{standing waves} is that they select preferred wavelengths ($\lambda$), and frequency $\omega$, and this wave becomes dominant in the medium. \\\\

Lets think of a an example, when we pluck a guitar string it oscillates, and this is true for waves! So, when we plunk this string it will hit the end of a medium, lets call this a node, and when the wave reaches the node it will reflect back on itself.

\begin{center}
    \includegraphics*[scale = .2]{imgs/standing-wave-diagram.jpeg}
\end{center}

Now, lets analyze what this would be if we have a wavelength inside of this cavity. We see that we have destructive interference at these nodes, and we have constructive interference between these nodes.

\begin{center}
    \includegraphics*[scale = .15]{imgs/standing-wave-with-nodes.jpeg}
\end{center}

\newpage

\begin{center}
    \includegraphics*[scale = .75]{imgs/book-standing-wave-diagram.png}
\end{center}

Above is a diagram pulled from the book to show us the equation of the resulting \textbf{standing wave}. We see that \textbf{constructive interference} happens at $t = \frac{T}{2}, \frac{3T}{2}, \dots$, and we get \textbf{destructive interference} at $t = 0, T, 2T, \dots$ \textit{Note: $\varphi_R$ is a phase shift of $\pi$ upon reflection. Also, where constructive interference happens, this is called an anti-node.}

\newblock

Let's say that the phase shift due to reflection ($\varphi_R$) is not equal to $\pi$, then the node position is shift, but the nodes are still separated by half of a wavelength ($\frac{\lambda}{2}$), and the anti-nodes are shifted too! This is still a standing wave.

\newblock

An important idea behind \textbf{standing waves} is that they transmit no energy, unlike traveling waves. Another example of standing waves are lasers! Lasers use standing waves, by forming a cavity between two mirrors to send a wave back and forth like the image below.

\begin{center}
    \includegraphics*[scale = .75]{imgs/standing-wave-for-lazer.png}
\end{center}

Looking at the diagram above we see that only wavelengths with descrete values like a inteeger number of half-wavelengths will fit into the distance between the two mirrors, which we will label $d$.

\[d = m \bigg( \frac{\lambda\cdot m}{2} \bigg)\]

\newpage

We can also find the the wavelength of the standing wave modes by: 

\[\lambda_m = \frac{2d}{m}\]

Also, we can find the frequencies of the standing wave modes ($\nu_m$):

\[\nu_m = \frac{v}{\lambda} \to \text{ we know that the wave is light so $v = c$, so}\]
\[ \nu_m = \frac{v}{\lambda} = \frac{c}{\lambda} = \frac{c\cdot m}{2d}\]

Note that these equations are only valid for cavities with plane mirrors.

\textbf{Example} 
\\ \newblock
A certain He-Ne laser cavity of the type shown in Figure 6 has a mirror
separation of 30 cm. The helium-neon laser gain medium is capable of
supporting laser light of wavelengths in the range from $\lambda_1 = 632.8$nm to $\lambda_2 = 632.802$nm, find:
\\ \newblock
a. The approximate number $m$ of half-wavelengths that fit into the cavity

b. The range of frequencies supported by the helium-neon gain medium

c. The difference in the frequencies of adjacent standing wave modes of
the cavity

d. The number of standing wave modes that will likely be present in the
laser output

\begin{center}
    \includegraphics*[scale = .2]{imgs/cavity-example.jpeg}
\end{center}

\newpage

\begin{center}
    \includegraphics*[scale = .2]{imgs/cavity-example-cont.jpeg}
\end{center}

\question{Determine the beat frequency of two waves}







\end{document}